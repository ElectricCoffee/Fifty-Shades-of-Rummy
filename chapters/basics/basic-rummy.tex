\newpage\section{Basic Rummy}
The following section outlines the most basic version of Rummy that I can think of, which uses most of the stuff detailed above.

\subsection{Requirements}
\begin{description}
    \item[Number of Players:] 2--6. 
    \item[Cards:] A standard deck of 52 playing cards (No Jokers).
    \item[Cards Dealt:] 7.
    \item[Requirements for Going Out:] None.
\end{description}

\subsection{Setup}
\begin{enumerate}
\item Decide on an amount of points to play to. 100 is a common choice.
\item Shuffle the cards thoroughly and deal them one at a time to each player, until everyone has 7 cards.
\item Place the remainder of the cards in the centre of the table to form the supply.
\item Turn over the top card and set it next to the supply to form the discard.
\item The game is now ready to begin, starting with the player left of the dealer.
\end{enumerate}
\subsection{Gameplay}
Every turn has the same basic structure:
\begin{enumerate}
    \item Draw a card.
    \item Play any number of melds.
    \item Lay off any number of cards.
    \item Discard a card if possible.
\end{enumerate}

\subsubsection{Drawing Cards}
At the start of your turn, you have two choices when it comes to drawing cards: you can either pull the topmost card from the supply, or the topmost one face up in the discard.
Note that this step is \textit{not} optional.
\subsubsection{Playing Melds}
After drawing a card, you may play any number of melds from your hand. Valid melds include \textit{Sets} of 3--4 cards of the same rank and \textit{Runs} of 3 or more sequential cards in the same suit.

For the purposes of sets, aces are considered to occur after kings, meaning $\clubsuit Q, \clubsuit K, \clubsuit A$ is a valid set, but $\diamondsuit A, \diamondsuit 2, \diamondsuit 3$ is not.
\subsubsection{Laying off Cards}
After playing your first meld, you can begin laying off cards to any of the existing melds on the table, including those of your opponent.
The goal of the game is to get rid of your cards after all, so dumping your unwanted cards onto any meld you can see is beneficial.
\subsubsection{Discarding a Card}
At the end of your turn---if you have any cards left---you discard a card onto the discard pile, and give the turn to the player to the left of you.
\subsubsection{Going Out}
If playing a meld, laying off cards, or discarding a card would leave you with an empty hand, you've successfully \textit{gone out}, and the round is immediately over.

\subsubsection{Declaring Rummy}
If your hand consists entirely of valid melds, and you have neither played melds nor laid off any cards in any of your turns, then you can choose to \textit{declare Rummy}. To do so, say the word ``Rummy'' out loud and place your entire hand onto the table for scrutiny, optionally discarding a card as normal.

If declared valid by everyone at the table, your effective score for the round is doubled.

\subsection{Scoring}
Once a player goes out it's time to score.
The winner counts up all the cards left over in the other players' hands.
Whichever player first reaches the agreed-upon amount of points wins the game.

The cards are valued as follows:
\begin{center}
    \begin{tabular}{rcl}
        2--10 & = & Printed value\\
        J,Q,K & = & 10\\
        A & = & 15
    \end{tabular}
\end{center}
