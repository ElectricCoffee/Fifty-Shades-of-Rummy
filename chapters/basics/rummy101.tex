\section{Rummy 101}
Most variations of Rummy will see you do the same things: 
\begin{enumerate}
    \item Form melds of sets and/or runs
    \item Get rid of all your cards
\end{enumerate}
The things that usually sets these games apart are rules surrounding
\begin{itemize}
    \item Number of cards dealt
    \item Number of players
    \item Scoring criteria
    \item Melding criteria
    \item etc.
\end{itemize}
This may all seem like a lot, but don't worry, we're going to get through this one step at a time.

Most varieties of Rummy share the same core elements: a supply, a discard pile, melds.

\subsection{Card/Tile}
Rummy games are often played with either cards (most variants) or tiles (Rummikub and Mahjong), and usually carry two types of markings: a rank and a suit.

\subsection{Rank}
The rank of a card or tile is the printed---often numeric---value. $\heartsuit Q, \spadesuit Q$ for example, share the same rank, but not the same suit.

\otherNames \textit{(Face) Value}, \textit{Number}.

\subsection{Suit}
The suit of a card or tile is the \textit{other} primary distinguishing feature on a card or tile. $\diamondsuit 4, \diamondsuit K$ both share the same suit, but not the same rank.

\otherNames \textit{Colour}.

\subsection{Hand}
Your \textit{Hand} are the cards or tiles you hold onto yourself. Be they seated on a rack or held in your own hand, these are the cards or tiles you draw into, build melds out of, and discard from.

\subsection{Supply}
The supply is a stack or pile of face-down cards or tiles in the middle of the table.
The supply serves as the primary means of attaining new cards or tiles.

\otherNames \textit{Draw Pile}, \textit{Wall}, \textit{Pouch}, \textit{Stock}.
\subsection{Discard}
Present in most Rummy variants, the discard pile is where you get rid of the cards or tiles you do not wish to keep or use to build melds out of.

These discarded cards or tiles then become available for the other players to draw upon, though usually with certain restrictions.
\otherNames \textit{Graveyard}, \textit{Dump}, \textit{Waste}.
\subsection{Meld}
Your primary means of winning, and the one defining attribute of Rummy, a meld is any valid combination of cards or tiles, which are used to empty your hand of cards or tiles.

Melds usually come in the forms of \textit{Sets} and \textit{Runs}, and on rare occasions \textit{Pairs}.

\otherNames \textit{Phase} (Phase 10), \textit{Set} (Rummikub).
\subsection{Set}
One of the two must common types of melds, a set is a group of three or more cards or tiles with the same value. $\heartsuit 2, \spadesuit 2, \diamondsuit 2$ is a typical example of a set.

Restrictions on what constitutes a valid set varies from variant to variant, but sharing the same rank is always a requirement.
\otherNames \textit{Group}, \textit{Book}, \textit{Pon/Pung}, \textit{Kan/Kong}.
\subsection{Run}
The other common type of meld, a run is a group of three or more cards or tiles of consecutive rank, usually also sharing the same suit.
$\heartsuit 2, \heartsuit 3, \heartsuit 4$ is an example of a run of three consecutive numbers, sharing the suit of $\heartsuit$.
\paragraph{Other Names:} \textit{Straight}, \textit{Sequence}, \textit{Chii/Chow}.

\subsection{Pair}
Though uncommon, as it is only used in very few variants, the pair still bears mention.
A pair is merely set containing only two cards or tiles.

\subsection{Going Out}
Going out happens at the very end when you either play your last meld or \textit{show} your fully melded hand.

It is usually accompanied with saying the name of the game out-loud to alert the other players of the round ending.

Some variants have special requirements for \textit{going out}. Going out means getting rid of the last card in your hand and finishing the round.

The requirements for going out typically involve either having melds totalling a certain value, or having a specific meld or combination of melds either in your hand or on the table.

\otherNames \textit{Show}.

\subsection{Scoring}
Most variants have some form of scoring, though the rules for which vary wildly from version to version, though more often than not, it involves counting up the values in your opponents' hands at the end of the game.